\section{UMTS}
    \begin{tabular}{|l|l|l|}
        \hline
        
        & \textbf{UMTS}
        & \textbf{GSM}\\
        \hline
        Max. Sendeleistung
        & 125-250\,mW
        & 2\,W \\
        \hline
        Kanalzugriffsverfahren
        & CDMA
        & TDMA \\
        \hline
        Datenrate/Teilnehmer
        & bis 2\,Mbit/s 
        & 9.6\,kBit/s \\
        \hline
        Bandbreite/Kanal
        & 5\,MHz
        & 200\,kHz \\
        \hline
        Pulsfrequenz
        & 100\,Hz
        & 217\,Hz \\
        \hline
        Zeitschlitze/Frame
        & 15
        & 8 \\
        \hline
        Frequenzband
        & 2\,GHz
        & 900\,MHz / 1800\,MHz \\
        \hline
        Max. Sprachverbindungen / Kanal
        & 108
        & 8 \\
        \hline
        Max. Zellenradius
        & ca. 8\,km
        & 35\,km \\
        \hline
    \end{tabular}

\subsection{Nichtionisierende Strahlung (NIS/NIR) \formelbuch{253}}
Im Gegensatz zu ionisierender Strahlung (Röntgenstrahlung) kann nichtionisierende Strahlung keine Moleküle 
in ihrer Struktur verändern, jedoch sind thermale Effekte (Erwärmung der Körperzellen) bestätigt.

\begin{center}
\begin{tabular}{|l|l|l|l|} \hline
    & SAR-Wert
    & GSM 900
    & GSM 1800 \\
    \hline    
        \textit{Vergleichswert}
        & \multicolumn{3}{l|}{4\,W/kg $\Rightarrow$ 1-2$^\circ$C Erwärmung nach 30min} \\       
    \hline    
    Professionelle Aussetzung  
        & \multicolumn{3}{l|}{0.4\,W/kg (Techniker während Arbeit in Anlagen)} \\      
    \hline
    Internationale Grenze  
        & 0.08\,W/kg
        & 42\,V/m
        & 58\,V/m \\
    \hline
    Schweizer Grenze 
        & 800\,$\mu$W/kg
        & 4\,V/m
        & 6\,V/m \\
	\hline
	\end{tabular} \\
	NIS Grenzwerte für Installationen im Vergleich
	\end{center}

